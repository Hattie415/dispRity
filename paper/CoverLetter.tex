\documentclass[11pt]{letter}
\usepackage[a4paper,left=2.5cm, right=2.5cm, top=1cm, bottom=1cm]{geometry}
\usepackage[osf]{mathpazo}
\signature{Thomas Guillerme}
\address{Imperial College London \\Silwood Park Campus \\Buckhurst Road \\Ascot SL57PY, United Kingdom \\guillert@tcd.ie}
\longindentation=0pt
\begin{document}

\begin{letter}{}
\opening{Dear Editors,}

Studying multivariate data as an alternative to univariate data has become increasingly popular in the last decades both in ecology and evolution.
This includes, for example, studying form and function (e.g. Diaz et al., 2016, \textit{Nature}) or morphological diversity (i.e. disparity - e.g. Brusatte et al. 2012, \textit{Science}).
These methods have in common to use a multidimensional space (through ordination or other techniques) representing the combination of all possible observed traits and then measuring some aspect of the space (e.g. its hyper-volume) for different groups and compare them between each other or to a null hypothesis.

Hence there exist a multitude of ways to define a multidimensional space, a metric measuring a specific aspect thereof and a test for many hypothesis.
In theory this vast tool-kit should allow evolutionary biologists and ecologists to do many things.
There are numerous implementation of such analysis pipeline in numerous R packages (e.g. \texttt{Claddis}, \texttt{geomorph}, \texttt{ade4}, \texttt{vegan}, etc.) or even in other languages (e.g. \texttt{GINGKO} in javascript).
However, each of these implementations are particular to the software/package author's definition of the multidimensional space, metric and test.
This leads researchers to often use these tool-kits by default hampering them to use the best definitions of the space/metric/test fitted to their particular question.
At worst this leads in turn to wrong statistic claims, and, at best to time consuming code retro-engineering.

In this application manuscript entitled ``dispRity: an R package for measuring disparity'', I present a complete and flexible R package allowing to run multidimensional analysis where each parameter (the space, the metric and the test) can be easily defined by the user.
In fact, this package provides a tidy and modular environment where users can use their own multidimensional space defined as a matrix and apply to it any metric and any test (provided as functions).
This allows for highly modular multidimensional analysis and the tidy interface allows easily reproducible analysis.

I look forward to hearing from you soon,

\closing{Yours sincerely,}

\end{letter}
\end{document}


%Editors:
%Sam Price, Daniele Silvestro

%Reviewers
%Graeme Lloyd, Dave Bapst, Emma Sherratt 
