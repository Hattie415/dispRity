\documentclass[11pt]{letter}
\usepackage[a4paper,left=2.5cm, right=2.5cm, top=1cm, bottom=1cm]{geometry}
\usepackage[osf]{mathpazo}
\signature{Thomas Guillerme}
\address{Imperial College London \\Silwood Park Campus \\Buckhurst Road \\Ascot SL57PY, United Kingdom \\guillert@tcd.ie}
\longindentation=0pt
\begin{document}

\begin{letter}{}
\opening{Dear Editors,}

Studying multivariate data as an alternative to univariate data has become increasingly popular in the last decades in both ecology and evolution.
This includes, for example, studying form and function (e.g. Diaz et al., 2016, \textit{Nature}) or morphological diversity (i.e. disparity - e.g. Brusatte et al. 2012, \textit{Science}).
These methods all use a multidimensional space (through ordination or other techniques) representing the combination of all possible observed traits and then measuring some aspect of the space (e.g. its hypervolume) for different groups, and comparing these to each other or to a null hypothesis.

There are a multitude of ways of carrying out each of these steps, and numerous software implementations in several languages (e.g. in R: \texttt{Claddis}, \texttt{geomorph}, \texttt{ade4}, \texttt{vegan}, or in javascript: \texttt{GINGKO}).
However, each of these implementations is particular to the package author's definition of the multidimensional space, metric and test.
This leads researchers to use these toolkits by default, thus preventing them from using the best definitions of the space/metric/test for their particular question or dataset.
At worst this leads to incorrect statistic claims, and, at best to time consuming code retro-engineering.

In this application manuscript entitled ``dispRity: an R package for measuring disparity'', I present a complete and flexible R package for running multidimensional analysis where each parameter can be easily defined by the user (the space, the metric and the test).
This package provides a tidy and modular environment where users can use their own multidimensional space defined as a matrix and apply to it any metric and any test (provided as functions).
This allows for highly modular multidimensional analysis and the tidy interface allows easily reproducible analysis.
All functions have documentation and unit tests. 
Additionally, this package comes with an in depth Gitbook manual covering most functionalities that will be updated regularly following user requests. 
Finally, the package is likely to be well cited as it is already being used in at least five projects that I am aware of.

I look forward to hearing from you soon,

\closing{Yours sincerely,}

\end{letter}
\end{document}


%Editors:
%Sam Price, Daniele Silvestro

%Reviewers
%Graeme Lloyd, Dave Bapst, Emma Sherratt 
